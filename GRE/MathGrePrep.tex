\documentclass[a4paper]{article}
\usepackage[utf8]{inputenc}

\usepackage[margin=0.7in]{geometry}  % Default margin
\usepackage{mathpazo}                % Changes font to mathpazo :^)
\usepackage{amsmath,amssymb}         %
\usepackage{fancyhdr}                %
\usepackage{amsthm}

\pagestyle{fancy}                    % These are used for custom headers, use
\lhead{\bfseries \name}              % \def\name{name} and \def\assigment{assign}
\chead{\bfseries \assignment}        % in the hw latex file to assign values
\rhead{\bfseries \thepage}           %
\cfoot{}                             %

\newcommand{\Q}{\mathbb{Q}}          % 
\newcommand{\C}{\mathbb{C}}          %
\newcommand{\Z}{\mathbb{Z}}          %
\newcommand{\re}{\text{Re}}          %
\newcommand{\im}{\text{Im}}          %
\newcommand{\pypx}[2]{\frac{\partial #1}{\partial #2}}
\newcommand{\Log}{\text{Log}}        %
\newcommand{\Arg}{\text{Arg}}        %

\setlength{\parindent}{0pt}          % Sets default paragraph indent length to 0

\newenvironment{problem}[2]{\\[10pt] \textbf{#1. } \textit{#2\\[10pt]}\indent}{}

\def\name{Daniel Tobias}
\def\assignment{Mathematics GRE Review}
\usepackage{hyperref}
\begin{document}
\section{Introduction}
This document will serve as a review for the Mathematics Subject GRE. I am planning on taking the test in October and 2019, and I plan on reviewing (nearly) all of the topics on the GRE. My background is in pure mathematics, particularly I enjoy abstact algebra, mathematical logic, and computational complexity. However I have noticed that spending a lot of time away from things like calculus ultimately has slowed me down to unacceptable levels. What follows in the document will be completed exercises, commentary, and thoughts about the GRE. Hopefully someone may find this useful.\\

To begin, ETS gives us a set of topics to review and their frequencies on the GRE. It makes sense to predetermine the order of which I go through these topics and decide which resources I would like to use for them. Note that this document is being updated as I go through this and certain review decisions may completely change.

\begin{enumerate}
	\item \textbf{Calculus - 50\%} In this section we will cover the basics of calculus up to Real Analysis. The plan is to cover \cite{Stewart} for calculus, \cite{Rudin} for analysis (the early chapters), and cover some calculus problems from the Princeton Review and \cite{UC} tests.
\end{enumerate}

\section{Calculus}

\begin{thebibliography}{9}
	\bibitem{UC}
	University of Chicago. Math GRE Preparation Materials. \href{https://math.uchicago.edu/~min/GRE/}{Link}.
	\bibitem{Rudin}
	Rudin, Walter. Principles of Mathematical Analysis. \href{https://notendur.hi.is/vae11/\%C3\%9Eekking/principles_of_mathematical_analysis_walter_rudin.pdf}{Link}.
	\bibitem{Stewart}
	Stewart, James. Calculus. Physical Copy.
\end{thebibliography}

\end{document}
