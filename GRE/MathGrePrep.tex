\documentclass[a4paper]{article}
\usepackage[utf8]{inputenc}

\usepackage[margin=0.7in]{geometry}  % Default margin
\usepackage{mathpazo}                % Changes font to mathpazo :^)
\usepackage{amsmath,amssymb}         %
\usepackage{fancyhdr}                %
\usepackage{amsthm}

\pagestyle{fancy}                    % These are used for custom headers, use
\lhead{\bfseries \name}              % \def\name{name} and \def\assigment{assign}
\chead{\bfseries \assignment}        % in the hw latex file to assign values
\rhead{\bfseries \thepage}           %
\cfoot{}                             %

\newcommand{\Q}{\mathbb{Q}}          % 
\newcommand{\C}{\mathbb{C}}          %
\newcommand{\Z}{\mathbb{Z}}          %
\newcommand{\re}{\text{Re}}          %
\newcommand{\im}{\text{Im}}          %
\newcommand{\pypx}[2]{\frac{\partial #1}{\partial #2}}
\newcommand{\Log}{\text{Log}}        %
\newcommand{\Arg}{\text{Arg}}        %

\setlength{\parindent}{0pt}          % Sets default paragraph indent length to 0

\newenvironment{problem}[2]{\\[10pt] \textbf{#1. } \textit{#2\\[10pt]}\indent}{}

\def\name{Daniel Tobias}
\def\assignment{Mathematics GRE Review}
\usepackage{hyperref}
\begin{document}
\section{Introduction}
This document will serve as a review for the Mathematics Subject GRE. I am planning on taking the test in October of 2019, and I plan on reviewing (nearly) all of the topics on the GRE. My background is in pure mathematics, particularly I enjoy abstact algebra, mathematical logic, and computational complexity. However I have noticed that spending a lot of time away from things like calculus ultimately has slowed me down to unacceptable levels. What follows in the document will be completed exercises, commentary, and thoughts about the GRE. Hopefully someone may find this useful.\\

To begin, ETS gives us a set of topics to review and their frequencies on the GRE. It makes sense to predetermine the order of which I go through these topics and decide which resources I would like to use for them. Note that this document is being updated as I go through this and certain review decisions may completely change.

\begin{enumerate}
	\item \textbf{Calculus - 50\%} In this section we will cover the basics of calculus up to Real Analysis. The plan is to cover \cite{Stewart} for calculus, \cite{Rudin} for analysis (the early chapters), and cover some calculus problems from the Princeton Review and \cite{UC} tests.
\end{enumerate}

\section{Pre-Calculus}
\subsection{Trigonometric Functions}
In mathematics it sometimes helps us to derive results to better understand where they come from. This in turn anchors the facts in our memory. Unfortunately, trigonometric identities do not seem to behave in this way. However, I will derive them now anyways using the exponential function.
\subsubsection{Some Identities}
\begin{align}
	e^{i\theta} &= \cos(\theta) + i\sin(\theta)\\
	e^{i\theta}e^{i\tau} &= (\cos(\theta) + i\sin(\theta))(\cos(\tau)+i\sin(\tau))\\
	e^{i(\theta+\tau)} &= \cos(\theta)\cos(\tau)-\sin(\theta)\sin(\tau)+i(\sin(\theta)\cos(\tau) + \sin(\tau)\cos(\theta))
\end{align}
Letting $\theta = \tau$ will provide us with some useful identities:
\begin{align}
	e^{2i\theta} = \cos^{2}(\theta)-\sin^{2}(\theta)+i(2\sin(\theta)\cos(\theta))
\end{align}
Taking real and imaginary parts respectively,
\begin{align}
	\Re(e^{2i\theta}) &= \cos(2\theta) = \cos^{2}(\theta) - \sin^{2}(\theta)\\
	\Im(e^{2i\theta}) &= \sin(2\theta) = 2\sin(\theta)\cos(\theta)
\end{align}
Knowing that $-\sin^{2} = \cos^{2} - 1$ we can now substitute into (5) to get
\begin{align}
	\cos(2\theta) = 2\cos^{2}(\theta) - 1 \quad \iff \quad \cos^{2}(\theta) = \frac{1 + \cos(2\theta)}{2}\\
\end{align}
Similarly letting $\cos^2 = 1 - \sin^2$ we have
\begin{align}
	\cos(2\theta) = 1 - 2\sin^{2}(\theta) \quad \iff \quad \sin^{2}(\theta) = \frac{1-\cos(2\theta)}{2}
\end{align}
We will find in calculus that is useful to have this ability to reduce the power of trigonometric functions, particularly as an integration technique.
The formulas which we just derived are known as the \textit{double angle formulas}. There are also \textit{half angle formulas} by taking $\theta = \frac{\tau}{2}$ and taking a square root.
Continuing on our identity frenzy it makes sense to talk about the \text{sum and addition formulas} which are closed form expressions in the case of $e^{i (\theta \pm \tau)}$. We leave this as an exercise to derive, and they also somewhat rely on properties we will mention in the following section.

\subsubsection{Extension to $\C$}
\begin{align*}
	e^{iz} + e^{-iz}= \cos(z) + i\sin(z) + \cos(z) - i\sin(z) = 2\cos(z) \quad \iff \quad \frac{e^{iz}+e^{-iz}}{2} = \cos(z)\\
	e^{iz} - e^{-iz} = \cos(z) + i\sin(z) - \cos(z) + i\sin(z) = 2i\sin(z) \quad \iff \quad \frac{e^{iz}-e^{-iz}}{2i} = \sin(z)
\end{align*}
\subsubsection{Some Properties of Trigonometric Functions}
We state the following identities without proof.
\begin{align*}
	&\cos(-x) = \cos(x)\\
	&\sin(-x) = -\sin(x)\\
	&\cos(x + 2\pi) = \cos(x)\\
	&\sin(x + 2\pi) = \sin(x)\\
	&\cos\left(x - \frac{\pi}{2}\right) = \sin(x)\\
	&\sin\left( x + \frac{\pi}{2}\right) = \cos(x)
\end{align*}
And since we've been ignoring $\tan$ for some time
\begin{align*}
	\tan(\theta + \tau) = \frac{\tan(\theta)+\tan(\tau)}{1-\tan(\theta)\tan(\tau)}
\end{align*}
We can see easily how negating $\tau$ would yield another identity. In terms of power reduction we have the following formula 
$$\tan(2\theta) = \frac{2\tan(\theta)}{1-\tan^{2}(\theta)}$$\\

\subsubsection{Vieta's Formulas}
An extremely useful trick we have for finding roots of polynomials are \textit{Vieta's Formulas}.
Say we have a polynomial $a_n x^n + a_{n-1}x^{n-1} + ... + a_{0}$ with roots $r_{1}, r_2, ..., r_n$.
Then $$a_{n}x^n + a_{n-1}x^{n-1} + ... + a_0 = a_{n}(x-r_1)(x-r_2)\cdots(x-r_n)$$
It becomes clear from multiplying all of the constant terms that $a_{n}\prod_{i=0}^{n}r_i = (-1)^{n}a_0$ in other words $$r_1 r_2 \cdots r_n = (-1)^{n}\frac{a_0}{a_n}$$
Similarly by counting coefficients (particularly of $x$)$$r_0 + r_1 + ... + r_n = (-1)^{n} \frac{a_{n-1}}{a_n}$$

\begin{problem}{Example 1}{Assuming all roots of $x^3 - 3x^2 + 6x - 4$ are of the form $1+bi$, find the maximum value of $b$}
	First we note that this polynomial is of degree 3, and it is not possible that it has all real roots.
	Thus the roots we are looking for are $1+bi$, $1-bi$, and $r$. By Vieta $$1+bi + 1 -bi + r = 3 \rightarrow r = 1$$
	$$(1+bi)(1-bi) = 4 \rightarrow b = \sqrt{3}$$
\end{problem}

\subsubsection{Problems}
\textbf{1.} If $x \in \R$ and $\sin\sin(x) = \frac{1}{2}$ with $2 < x < 3$. Compute $\cos(-\sin(x))$.\\

\textit{Solution.} The equation yields $\sin(x) = \arcsin\left(\frac{1}{2}\right) \Rightarrow \sin(x) = \frac{\pi}{6}$ taking $\cos$ of both sides 
yields $\cos(\sin(x)) = \frac{\sqrt{3}}{2}$.\\\\

\textbf{2.} Define $\sinh(x) = \frac{e^{x}-e^{-x}}{2}$, what is a formula for $\sinh^{-1}(x)$?\\

\textit{Solution.} Let $y = \sin^{-1}(x)$, we wish to solve $x = \frac{e^{y}-e^{-y}}{2}$:
\begin{align*}
	2x &= e^{y}-e^{-y}\\
	2xe^{y} &= e^{2y} - 1 \Rightarrow 0 = e^{2y} - 2xe^{y} - 1 \quad let t = e^{y}\\
	0 &= t^2 - 2xt - 1 \Rightarrow t = x \pm \sqrt{x^2 + 1}\\
	y &= \log(x \pm \sqrt{x^2 + 1})
\end{align*}

\textbf{3.} For which value $\theta$ is $\frac{2+3i\sin(\theta)}{1-2\sin(\theta)}$ purely imaginary?\\

\textit{Solution.} Multiply the expression by the complex conjugate of the denominator to yield:
\begin{align*}
	\frac{2 - 6\sin^{2}(\theta)+7i\sin^{2}(\theta)}{1 + 4\sin^{2}(\theta)}
\end{align*}
It is clear that there is a lot of extraneous information in this expression, we need only that the real parts 'cancel out'.
\begin{align*}
	2 &= 6\sin^{2}(\theta)\\
	\sqrt{\frac{1}{3}} &= \sin(\theta) \Rightarrow \theta = \arcsin\left(\sqrt{\frac{1}{3}}\right)
\end{align*}

The problems seen here are not particularly interesting. The use of trigonometry is to
reduce the complexity of problems in the future for the most part.
\begin{thebibliography}{9}
	\bibitem{UC}
	University of Chicago. Math GRE Preparation Materials. \href{https://math.uchicago.edu/~min/GRE/}{Link}.
	\bibitem{Rudin}
	Rudin, Walter. Principles of Mathematical Analysis. \href{https://notendur.hi.is/vae11/\%C3\%9Eekking/principles_of_mathematical_analysis_walter_rudin.pdf}{Link}.
	\bibitem{Stewart}
	Stewart, James. Calculus. Physical Copy.
	\bibitem{ptest}
	GRE Practice Test. \href{https://www.ets.org/s/gre/pdf/practice_book_math.pdf}{Link}. 
\end{thebibliography}

\end{document}
