\documentclass[a4paper]{article}
\usepackage[utf8]{inputenc}

\usepackage[margin=0.7in]{geometry}  % Default margin
\usepackage{mathpazo}                % Changes font to mathpazo :^)
\usepackage{amsmath,amssymb}         %
\usepackage{fancyhdr}                %
\usepackage{amsthm}

\pagestyle{fancy}                    % These are used for custom headers, use
\lhead{\bfseries \name}              % \def\name{name} and \def\assigment{assign}
\chead{\bfseries \assignment}        % in the hw latex file to assign values
\rhead{\bfseries \thepage}           %
\cfoot{}                             %

\newcommand{\Q}{\mathbb{Q}}          % 
\newcommand{\C}{\mathbb{C}}          %
\newcommand{\Z}{\mathbb{Z}}          %
\newcommand{\re}{\text{Re}}          %
\newcommand{\im}{\text{Im}}          %
\newcommand{\pypx}[2]{\frac{\partial #1}{\partial #2}}
\newcommand{\Log}{\text{Log}}        %
\newcommand{\Arg}{\text{Arg}}        %

\setlength{\parindent}{0pt}          % Sets default paragraph indent length to 0

\newenvironment{problem}[2]{\\[10pt] \textbf{#1. } \textit{#2\\[10pt]}\indent}{}

\def\name{Daniel Tobias}
\def\assignment{Introduction to Mathematical Logic}

\begin{document}
\begin{problem}{1}{Prove there are no wffs of length 2, 3, or 6 but all other numbers work}
        We define the formula building operations

\begin{align}
    &\varepsilon_{\lnot}(\alpha) = (\lnot \alpha)\\
    &\varepsilon_{\land}(\alpha,\beta) = (\alpha \land \beta)\\
    &\varepsilon_{\lor}(\alpha, \beta) = (\alpha \lor \beta)\\
    &\varepsilon_{\rightarrow}(\alpha,\beta) = (\alpha \to \beta)\\
    &\varepsilon_{\leftrightarrow}(\alpha, \beta) = (\alpha \leftrightarrow \beta) 
\end{align}
We notice that if we start with 1 symbol, we can either add 3 symbols or 4. We can see then that we cannot possibly get the
values 2, 3, or 6 with these additions however 4 and 5 are possible. We then notice in general any length $n$ is possible given that $\exists x,y \in \mathbb{N}$ such that 
$$1 + 3x + 4y = n$$

\textbf{Lemma.} For coprime $a,b$ the equation $ax + by = c$ has natural solutions when $c \ge (a-1)(b-1)$.\\

\textit{Proof.} The Euclidean Algorithm can yield solutions for us such that $au -bv = 1$. Which means we can generate solutions to $ax + by = c$ with
$x = cu -bt$ and $y=at-cv$, indeed $a(cu-bt) + b(at-cv) = c$ for $t \in \Z$. Asking for
natural solutions means $$cu-bt > -1, \quad\quad at - cv > -1 $$
combining these inequalities yields $$\frac{cu}{b} + \frac{1}{b} > t > \frac{cv}{a} - \frac{1}{a}$$
Moving on:
\begin{align*}
    \frac{cu}{b} &+ \frac{1}{b} - \left(\frac{cv}{a} - \frac{1}{a}\right) = c \frac{au-bv}{ab} + \frac{1}{b} + \frac{1}{a}\\
    &= \frac{c}{ab} + \frac{1}{b} + \frac{1}{a} \ge 1 + \frac{1}{ab} > 1
\end{align*}
Note this only works when $\frac{c}{ab} \ge 1 - \frac{1}{b}-\frac{1}{a} + \frac{1}{ab}$ in other words, $c \ge (a-1)(b-1)$. But now we know the difference between these two numbers
is 1, meaning there is an integer between them. $\qed$\\

Now using the lemma it is clear that there are natural number solutions for all wffs of length $n \ge (4-1)(3-1) +1 = 7$.
\end{problem}
\end{document}