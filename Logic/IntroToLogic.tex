\documentclass[a4paper]{article}
\usepackage[utf8]{inputenc}

\usepackage[margin=0.7in]{geometry}  % Default margin
\usepackage{mathpazo}                % Changes font to mathpazo :^)
\usepackage{amsmath,amssymb}         %
\usepackage{fancyhdr}                %
\usepackage{amsthm}

\pagestyle{fancy}                    % These are used for custom headers, use
\lhead{\bfseries \name}              % \def\name{name} and \def\assigment{assign}
\chead{\bfseries \assignment}        % in the hw latex file to assign values
\rhead{\bfseries \thepage}           %
\cfoot{}                             %

\newcommand{\Q}{\mathbb{Q}}          % 
\newcommand{\C}{\mathbb{C}}          %
\newcommand{\Z}{\mathbb{Z}}          %
\newcommand{\re}{\text{Re}}          %
\newcommand{\im}{\text{Im}}          %
\newcommand{\pypx}[2]{\frac{\partial #1}{\partial #2}}
\newcommand{\Log}{\text{Log}}        %
\newcommand{\Arg}{\text{Arg}}        %

\setlength{\parindent}{0pt}          % Sets default paragraph indent length to 0

\newenvironment{problem}[2]{\\[10pt] \textbf{#1. } \textit{#2\\[10pt]}\indent}{}

\def\name{Daniel Tobias}
\def\assignment{Introduction to Mathematical Logic}

\begin{document}
\begin{problem}{1}{Prove there are no wffs of length 2, 3, or 6 but all other numbers work}
        We define the formula building operations

\begin{align}
    &\varepsilon_{\lnot}(\alpha) = (\lnot \alpha)\\
    &\varepsilon_{\land}(\alpha,\beta) = (\alpha \land \beta)\\
    &\varepsilon_{\lor}(\alpha, \beta) = (\alpha \lor \beta)\\
    &\varepsilon_{\rightarrow}(\alpha,\beta) = (\alpha \to \beta)\\
    &\varepsilon_{\leftrightarrow}(\alpha, \beta) = (\alpha \leftrightarrow \beta) 
\end{align}
We notice that the result of the operations use 4, 5, 5, 5, and 5 symbols respectively. If we add to our list any 
single symbol $\alpha$ this allows for a wff of length 1. We can see clearly then that we cannot have a wff of length 2 or 3. Since the smallest 
we can have is of length 1, and the next smallest is of length 4. Each of these operations add 3 new symbols to a wff, so the question becomes 
how can we split 6 into these additions. If we start with either 1, 4, or 5 symbols (which are all possible), we see that we cannot add multiples of 3, or 4
to any of these numbers to get a wff of length 6. For 7 we see that we can add 3 to a wff of length 4 and get a wff of length 7 (this
can be done by applying $\varepsilon_{\lnot}$ on $\alpha$ twice).
\end{problem}
\end{document}